\chapter{Background}


% TODO: Rewrite
In this chapter, we will introduce the basic concepts of the Internet Protocol
Suite, briefly describe its origin, semantics, and some of its protocols.
Furthermore, SME and the hardware it will run on will be introduced as a basis 
for the implementation.


\section{Internet Protocol Suite (TCP/IP)}
Internet Protocol Suite, better known as simply TCP/IP, is a conceptual
model providing end-to-end communication between computers. It consists of
a collection of protocols specifying the communication between multiple
Internet systems\cite{RFC1122}.  The very early research and development
on what would later become the Internet Protocol Suite began in the late 1960s
by the Defense Advanced Research Project Agency (DARPA), and was being
adopted by DARPA, as well as the public, since 1983\cite{DARPA_internet}.
Although the Internet Protocol Suite predates the newer, arguably more
refined Open Systems Interconnection (OSI) model, TCP/IP still
remains the popular choice in modern systems.  As opposed to OSI 7-layer
model\cite{X.200}, the collection of protocols in TCP/IP are organized
into 4 abstraction layers, each related to their scope of networking
involved.

\subsection{Link Layer}
The link layer is the lowest, bottom-most layer in the Internet Protocol Suite.
Link layer addresses methods and protocols operating on the link that the host
is physically connected to\footnote{Wireless connections are also included
under this category.}. Contrary to the OSI model, this lowest layer in TCP/IP
does not regard the standards and protocols of the physical mediums used (the
pin layout, voltages, cable specifications etc.), making TCP/IP hardware-independent.
As a result, TCP/IP can in theory be implemented on virtually any hardware
configuration, emphasizing the flexibility of the model.

\subsection{Internet Layer}
The internet layer mainly concerns itself with sending data from the source
network to the destination network. This seemingly simple task requires multiple
functions from the layer:
\begin{itemize}
    \item Addressing and identification
    \item Packet routing
    \item \emph{Basic} transmit diagnostic information
    \item Carrying data for various upper layer protocols
\end{itemize}


\subsection{Transport Layer}
The transport layer establishes end-to-end data transfer between hosts.
Protocols in the transport layer can provide additional services to the user,
such as reliability, ordering, error- and flow-control, application addressing
(port numbers), error-checking, and so on.\\
While it is possible to bypass the protocols in this layer on most modern
network stacks, the protocols in the transport layer provide such essential
and useful services that it hardly ever makes sense to implement in the
application layer.\\
While there are numerous protocols defined in the Transport Layer, perhaps the 
most well-known protocol in the stack is the Transmission Control Protocol (TCP).
Being one of the most used transport protocol for its reliability and congestion 
control systems, it is rightly justified to refer to the whole Internet Protocol
Suite as simply "TCP/IP". 


\subsection{Application Layer}
The application layer protocols are used by applications and services to
exchange information over the network. A few of the well-known application
layer protocols are the Hypertext Transfer Protocol (HTTP)\cite{RFC1945},
File Transfer Protocol (FTP)\cite{RFC0114}, and Simple Mail Transfer Protocol
(SMTP)\cite{RFC0788}.\\
This layer is usually implemented by the userspace applications themselves, and 
therefore are not strictly required to actually run a TCP/IP network.



\section{Hardware}
The networking stack is intended to be flexible enough to run on just about any 
configuration of hardware and software. However, this also means that it cannot 
depend on any major external components, such as an existing memory, a processor,
or any form of operating system. Fundamentally, not only the software-part of the
networking stack has to be implemented, but the hardware needs to be defined 
as well. This hardware should be self-contained enough to work well in combination 
with any additional system, which the user incorporate for networking.\\
A wide variety of hardware types exist for such independent system, such as 
Application-specific Integrated Circuit (ASIC), Complex Programmable Logic 
Device (CPLD), Socket on a Chip (SoC), and Field-Programmable Gate Array (FPGA).
Each of these integrated circuits have their advantages and disadvantages; some
of them are re-programmable, some are cheap and disposable, and some are excellent 
for general-purpose applications.\\
In this thesis, only FPGAs will be taken into consideration for its re-programmability,
its fairly low-cost, and the compatibility with SME code-generators.


\subsection{Field Programmable Gate Array (FPGA)}
Field Programmable Gate Arrays, or FPGA for short, are devices containing 
integrated circuits (ICs) consisting of arrays of logic blocks. 
These logic blocks can be programmed to form arbitrary logic circuit by simply 
synthesizing a design and then loading it onto the board. This process alone 
can save the manufacturer months by not having to fabricate a whole new IC. \\
FPGAs can be used for any computational tasks without the need of any additional
hardware. Usually, these devices are used for smaller, domain-specific tasks,
where the control over the hardware yields significant performance increases.
FPGAs are indeed very universal, and can be used in product-design, prototyping, 
as well as in final products. Products like car driver assistance 
systems\cite{xilinx_fpga_automotive}, audio decoders\cite{xilinx_fpga_audio}, 
or even internet search engines\cite{bing_search_fpga}  all utilize FPGAs to 
increase the performance, lower the electrical bill, and  boost the development 
potential. 


\subsubsection{Technical specifications}
Field Programmable Gate Arrays consists of a vast number of ICs, which can 
be reprogrammed at any time for a desired application or functionality\cite{ni_fpga},
making the devices very flexible and extensible, even after manufacturing.\\
These ICs are practically totally independent, and their logic within can be 
programmed and combined in virtually any way with other ICs. This, however,
poses a problem, as signals do not propagate through circuitry, immediately, but
rather, they have a slight delay. 
Sometimes, two events precede each other, while other times, events of distinct 
timings must occur simultaneously.
Since the order of events is critical for correct and expected execution in 
digital circuits, a digital clock is used to ensure everything runs in sync.
A clock in this context emits a series of pulses in a pre-determined and very 
precise interval. These pulses are used to control the execution of various 
elements in the circuitry.\\
When synthesizing to a FPGA, the compiler finds the longest code-path, it finds
the required circuitry to perform the calculation, and then it determines the 
minimal required time for the signals to propagate through the path. In this 
manner, the fastest possible clock can be found for that particular circuit.\footnote{
    Although many modern FPGAs consist of multiple regions which can have individual 
    clock-rates. While it is a demanding task to propagate signals across these
    boundaries, a performance increases can be gained.
}
With innovations and steady improvements in modern FPGAs, the circuitries within
the devices can be clocked at higher than 500 MHz\cite{xilinx_fpga}.   



\subsubsection{Programming an FPGA}
Unlike conventional processors with a very sequential nature, the logic blocks 
in FPGAs are truly parallel in nature. Given the right programming, an FPGA can
allocate dedicated sections of the chip for each independent subtask, enabling
the circuitry to perform numerous independent calculations at once\cite{ni_fpga}.
Unfortunately, this universality of FPGAs comes at a cost to their performance.
Whereas conventional processors are heavily optimise based on the predetermined 
circuitry, FPGAs programmers must ensure to utilize the parallel nature of the
device in order to secure best possible performance. Even worse, the FPGA must 
be programmed in such a way that all paths in the electrical wiring can be 
in any time-frame.\\
Due to this parallel nature of FPGAs, conventional programming languages are 
next to impossible to use. To define the behavior of an FPGA, Hardware 
Description Languages (HDL) are used. These programming languages are not easy
to learn without a good grasp of electrical engineering. Even with prior 
programming knowledge, the unusual approach to concurrency in these languages 
can be hard to understand for average developers.\\
To simplify the development process, most manufacturers offer predefined 
circuits along their FPGAs. These predefined circuits are more commonly known
as Intellectual Property (IP) cores, and can provide the hardware designers 
with pre-made circuitry for a wide variety of functionality. While most IPs 
provide the functionality of processors for testing on an FPGA, mp3 audio 
decoding or PCI bus interconnect can be obtained as well\cite{fpga_for_dummies}.



\section{Synchronous Message Exchange} 
The Synchronous Message Exchange
model (SME) is a messaging framework created in order to help model
hardware descriptions\cite{sme_for_hardware_designs}.  It was conceived
once the flaws of using Communicating Sequential Processes (CSP) was
identified during the modelling of a vector processor with CSP using
PyCSP\cite{PyCSP}.  It turned out that there is a major discrepancy
between the way data is propagated in hardware opposed to that of the
CSP model. While CSP does not pose any requirements on the communication
between processes, in digital hardware, all communication has to be
synchronized, driven by a clock. To combat this in the CSP model, a
global clock process needed to be implemented, which was connected to
all other processes. Additionally, latches had to be introduced in order
to not overwrite values during a cycle. This caused an explosion of both
channels and latches in the final design, making CSP a much less viable
framework for hardware modelling\cite{sme_for_hardware_designs}.

\subsection{The model} The SME model consists of only a few fundamental
concepts. Each SME model is a \textit{network} consisting of one or more
\textit{processes}. These processes do not share any memory or storage,
but are interconnected with \textit{busses}.  These busses are perhaps
the most interesting units in SME model, as they not only propagate
information between processes using the underlying \textit{channels},
but also introduce an implicit clock between the processes.\\

\subsection{Process execution flow} The execution flow of a process is
fairly simple, and relates very closely to that of the actual hardware. At
the beginning of a clock-cycle, the input-ports are read into the busses
they are connected to. Then, the process executes its "compute" stage, and
the results, if any, are written to the output-port, which will be read
by the following bus. A visualization of the execution flow can be seen
on figure \ref{fig:sme_clock}.  \pic{0.5}{background/sme_clock}{An
illustration of a typical SME clock-cycle}{fig:sme_clock}
It is important to note that although certain channels might be written earlier
than others in a process clock, the subsequent processes connected to said bus 
will first see the values change in the beginning of the next clock cycle.


\subsection{Using SME}
SME has undergone multiple iterations, reworks, and extensions. While it is still
under very active testing and development, its core functionalities and features
 are well-established and stable\cite{bus_centric_sme}.\\
SME has concurrent implementations in the C\# and Python languages, with promising
efforts to unify these under a common intermediate domain-specific language 
SMEIL\cite{smeil}. The C\# version has exhibited various advantages over the 
Python counterpart, such as the more error-prone strong typing system, which 
better reflects the functionality of the hardware, as well as making the code
more readable to the programmer. At the time of writing, the C\# implementation currently
enjoys the most recent features of the SME model, as it is being the most actively 
developed version. 

