\chapter{Background}

In this chapter, we will introduce the basic concepts of the Internet Protocol
Suite, briefly describe its origin, semantics, and its reference implementation
in the early BSD systems.
Furthermore, SME will be introduced as a basis for the implementation.


\section{Internet Protocol Suite (TCP/IP)}
Internet Protocol Suite, better known as TCP/IP, is a conceptual model providing
end-to-end communication between computers. It consists of a collection of
protocols specifying the communication between multiple Internet systems\cite{RFC1122}.
The very early research and development on the Internet Protocol Suite began in 
the late 1960s by the Defense Advanced Research Project Agency (DARPA),
and was being adopted by DARPA, as well as the public, since 1983\cite{DARPA_internet}. 
Although the Internet Protocol Suite predates the newer, arguably more refined
Open Systems Interconnection model (OSI model), TCP/IP still remains the popular
choice in modern systems.
As opposed to OSI 7-layer model\cite{X.200}, the collection of protocols in TCP/IP are 
organized into 4 abstraction layers, each related to their scope of networking 
involved.

\subsection{Link Layer}
The link layer is the lowest, bottom-most layer in the Internet Protocol Suite.
Link layer addresses methods and protocols operating on the link that the host
is physically connected to\footnote{Wireless connections are also included 
under this category.}. Contrary to the OSI model, this lowest layer in TCP/IP 
does not regard the standards and protocols of the physical mediums used (the
pin layout, voltages, cable specifications etc.), making TCP/IP hardware-independent.
As a result, TCP/IP can in theory be implemented on virtually any hardware
configuration, emphasizing the flexibility of the model.

\subsection{Internet Layer}
The internet layer mainly concerns itself with sending data from the source 
network to the destination network. This seemingly simple task requires multiple
functions from the layer:
\begin{itemize}
    \item Addressing and identification
    \item Packet routing
    \item \emph{Basic} transmit diagnostic information
    \item Carrying data for various upper layer protocols
\end{itemize}

\subsection{Transport Layer}
The transport layer establishes end-to-end data transfer between hosts.
Protocols in the transport layer can provide additional services to the user,
such as reliability, ordering, error- and flow-control, application addressing 
(port numbers), error-checking, and so on. 

\subsection{Application Layer}
The application layer protocols are used by applications and services to 
exchange information over the network. A few of the well-known application
layer protocols are the Hypertext Transfer Protocol (HTTP)\cite{RFC1945},
File Transfer Protocol (FTP)\cite{RFC0114}, and Simple Mail Transfer Protocol
(SMTP)\cite{RFC0788}.\\
This layer is usually implemented by the applications themselves, and therefore 
are not strictly required to actually run a TCP/IP network.





\section{Synchronous Message Exchange}

