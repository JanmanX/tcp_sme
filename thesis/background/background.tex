\chapter{Background}


% TODO: Rewrite
In this chapter, we will introduce the basic concepts of the Internet Protocol
Suite, briefly describe its origin, semantics, and its reference implementation
in the early BSD systems.
Furthermore, SME will be introduced as a basis for the implementation.





\section{Field Programmable Gate Array (FPGA)}
Field Programmable Gate Arrays, or FPGA for short, are devices containing 
integrated circuits (ICs) consisting of arrays of logic blocks. These ICs can 
be reprogrammed at any time for a desired application or functionality\cite{ni_fpga},
making the devices very flexible and extensible, even after manufacturing.\\
Unlike conventional processors with a very sequential nature, the logic blocks 
in FPGAs are truly parallel in nature. Given the right programming, an FPGA can
allocate dedicated sections of the chip for each independent subtask, enabling
the circuitry to perform numerous independent calculations at once\cite{ni_fpga}.
Unfortunately, this universality of FPGAs comes at a cost to their performance.
Whereas conventional processors are heavily optimise based on the predetermined 
circuitry, FPGAs programmers must ensure to utilize the parallel nature of the
device in order to secure best possible performance.\\
Still, with innovations and steady improvements in modern FPGAs, the devices 
can easily reach a clock higher than 500 MHz\cite{xilinx_fpga}.   




\section{Internet Protocol Suite (TCP/IP)}
Internet Protocol Suite, better known as simply TCP/IP, is a conceptual
model providing end-to-end communication between computers. It consists of
a collection of protocols specifying the communication between multiple
Internet systems\cite{RFC1122}.  The very early research and development
on what would later become the Internet Protocol Suite began in the late 1960s
by the Defense Advanced Research Project Agency (DARPA), and was being
adopted by DARPA, as well as the public, since 1983\cite{DARPA_internet}.
Although the Internet Protocol Suite predates the newer, arguably more
refined Open Systems Interconnection model (OSI model), TCP/IP still
remains the popular choice in modern systems.  As opposed to OSI 7-layer
model\cite{X.200}, the collection of protocols in TCP/IP are organized
into 4 abstraction layers, each related to their scope of networking
involved.

\subsection{Link Layer}
The link layer is the lowest, bottom-most layer in the Internet Protocol Suite.
Link layer addresses methods and protocols operating on the link that the host
is physically connected to\footnote{Wireless connections are also included
under this category.}. Contrary to the OSI model, this lowest layer in TCP/IP
does not regard the standards and protocols of the physical mediums used (the
pin layout, voltages, cable specifications etc.), making TCP/IP hardware-independent.
As a result, TCP/IP can in theory be implemented on virtually any hardware
configuration, emphasizing the flexibility of the model.

\subsection{Internet Layer}
The internet layer mainly concerns itself with sending data from the source
network to the destination network. This seemingly simple task requires multiple
functions from the layer:
\begin{itemize}
    \item Addressing and identification
    \item Packet routing
    \item \emph{Basic} transmit diagnostic information
    \item Carrying data for various upper layer protocols
\end{itemize}

\subsection{Transport Layer}
The transport layer establishes end-to-end data transfer between hosts.
Protocols in the transport layer can provide additional services to the user,
such as reliability, ordering, error- and flow-control, application addressing
(port numbers), error-checking, and so on.\\
While it is possible to bypass the protocols in this layer on most modern
network stacks, the protocols in the transport layer provide such essential
and useful services that it hardly ever makes sense to implement in the
application layer.


\subsection{Application Layer}
The application layer protocols are used by applications and services to
exchange information over the network. A few of the well-known application
layer protocols are the Hypertext Transfer Protocol (HTTP)\cite{RFC1945},
File Transfer Protocol (FTP)\cite{RFC0114}, and Simple Mail Transfer Protocol
(SMTP)\cite{RFC0788}.\\
This layer is usually implemented by the applications themselves, and therefore
are not strictly required to actually run a TCP/IP network.




\section{Synchronous Message Exchange} The Synchronous Message Exchange
model (SME) is a messaging framework created in order to help model
hardware descriptions\cite{sme_for_hardware_designs}.  It was conceived
once the flaws of using Communicating Sequential Processes (CSP) was
identified during the modelling of a vector processor with CSP using
PyCSP\cite{PyCSP}.  It turned out that there is a major discrepancy
between the way data is propagated in hardware opposed to that of the
CSP model. While CSP does not pose any requirements on the communication
between processes, in digital hardware, all communication has to be
synchronized, driven by a clock. To combat this in the CSP model, a
global clock process needed to be implemented, which was connected to
all other processes. Additionally, latches had to be introduced in order
to not overwrite values during a cycle. This caused an explosion of both
channels and latches in the final design, making CSP a much less viable
framework for hardware modelling\cite{sme_for_hardware_designs}.

\subsection{The model} The SME model consists of only a few fundamental
concepts. Each SME model is a \textit{network} consisting of one or more
\textit{processes}. These processes do not share any memory or storage,
but are interconnected with \textit{busses}.  These busses are perhaps
the most interesting units in SME model, as they not only propagate
information between processes using the underlying \textit{channels},
but also introduce an implicit clock between the processes.\\

\subsection{Process execution flow} The execution flow of a process is
fairly simple, and relates very closely to that of the actual hardware. At
the beginning of a clock-cycle, the input-ports are read into the busses
they are connected to. Then, the process executes its "compute" stage, and
the results, if any, are written to the output-port, which will be read
by the following bus. A visualization of the execution flow can be seen
on figure \ref{fig:sme_clock}.  \pic{0.5}{background/sme_clock}{An
illustration of a typical SME clock-cycle}{fig:sme_clock}
It is important to note that although certain channels might be written earlier
than others in a process clock, the subsequent processes connected to said bus 
will first see the values change in the beginning of the next clock cycle.


\subsection{Using SME}
SME has undergone multiple iterations, reworks, and extensions. While it is still
under very active testing and development, its core functionalities and features
 are well-established and stable\cite{bus_centric_sme}.\\
SME has concurrent implementations in the C\# and Python languages, with promising
efforts to unify these under a common intermediate domain-specific language 
SMEIL\cite{smeil}. The C\# version has exhibited various advantages over the 
Python counterpart, such as the more error-prone strong typing system, which 
better reflects the functionality of the hardware, as well as making the code
more readable to the progrmamer. At the time of writing, the C\# implementation currently
enjoys the most recent features of the SME model, as it is being the most activelly 
developed version. 

