\chapter{Implementation}
\label{chap:implementation}
In this chapter, the implementation of the network stack using the pipeline
design is outlined and described, the application of SME detailed and evaluated,
and lastly, the viability of the system on an FPGA is discussed.\\



\subsection{Interface Signal protocols}
\label{sec:interface_signal_protocol}
With the introduction of buffers between each parsing processes, a clear pattern
emerged. The layer-handling processes are responsible for numerous real-time tasks 
(parsing, sending, protocol-specific tasks, etc), while also limited by their 
fixed internal buffers. These processes are not always ready to receive input 
from preceding processes, while they at the same time must be able to write their
output to following processes immediatelly.\\
The buffers are a stark opposite, as their large internal block memories enable
them to buffer huge chunks of memory, while also being able to wait for the 
succeeding process to start reading.\\
With these two established scenarios, protocols for each can be proposed.  

\subsection{Buffer-Producer data transfer}

\subsection{Compute-Producer data transfer}

\subsection{Interface Control}
