\chapter{Introduction}
% 1. general introduction
This thesis describes the design and implementation of an efficient, high-speed
TCP/IP network stack intended to run on custom hardware where performance, responsiveness,
and throughput is crucial.\\

% 2. Explanation of specific problem
As is the trend with modern automation, computerization, and mechanization, new
devices are steadily invented to handle this increasing demand for data and
control.
With the ever-increasing sophistication of machines generating immense amount
of information, the data needs to be transmitted to numerous other machines for
further processing, or even simply storage. The most common and the most convenient
way of linking multiple devices together is using the internet, and its underlying
protocols. However, the networking stack supplied with most major operating
systems, while heavily optimised, suffers from considerable penalties due to
complexities of a standard computer architecture. For example, heavy network
traffic utilizes the computers' internal busses, utilizes the memory, and spend
precious \gls{cpu} clock-cycles with polling and interrupts. This
prevents the machine from using these resources for actual computing tasks.\\
These issues have been identified and solved by hardware manufacturers by
adopting dedicated \gls{nic}, which would employ
various techniques to offload the processing. One such offloading technique is
called the \gls{toe}, which usually takes care of the essential
parts of networking involved -- the \gls{ip} and the \gls{tcp}\cite{TCP_offload_dumb_idea}.\\
Modern hardware manufacturers can produce NICs boasting network throughput
speeds as high as 100 Gigabits\cite{xilinx_100g_nic}. Unfortunately, these cards
are highly specialized for certain applications, and even though they provide
basic programmability, they are rarely suitable for rapid prototyping of
applications and other custom hardware devices. Furthermore, each NIC manufacturer
has a diverse set of hardware with varying interfaces, making it hard to
combine, swap and test these cards.
Licensed software solutions in the form of IP blocks exist as well
\cite{microtronix_ip_cores}\cite{avnet_ip_cores}. Unfortunately,
these blocks are usually distributed as black-boxes of \gls{vhdl} code, which is
hard to maintain, and even harder to debug and extend\cite{opencores_mission}.\\

In this thesis, we bridge the gap between the blazingly-fast network offloading
devices and their more flexible and malleable software counterparts.\\
This networking stack is implemented in a fully self-contained fashion so that
it is completely independent of any other software running on the machine, while
utilizing the performance advantages gained from the lack of overhead in
conventional implementations.
The use of a high-level programming language in combination with the modern
\gls{sme} model makes the network stack a very versatile
implementation with ease of use, debugging, and even extension.

% * VHDL unmaintainable
% * SME better for simulation
% * IP blocks not open source (No code) + Hard to debug
% * Price



MORE TO COME!!


% 3. Brief review of existing solutions

% 4. Output of proposed solution

% 5. Summary
