\chapter{Conclusion}
\label{chap:conclusion}

The goal of this project was to implement a working networking stack using the
\gls{sme} framework. Given that the \gls{sme} framework is still fairly new and
under heavy development, it was desired to test viability and the performance
of \gls{sme} for such a project.\\

The implemented networking stack underwent a few design-changes during
development, partly due to the new \gls{sme} model, and partly due to
distributed memory model present in the hardware. The final pipelined design
provides a clean architecture with a share-nothing module division, closely
ressembling the 4 layers of the Internet Protocol Suite model.\\

Limited tests were performed on the networking stack by simulating the system
on a computer and passing network packets generated by the host computer.
Although only about 10 mio.\ clock-cycles were simulated, over $17283$ packets
were sent to the system, and all of the $1280$ legitimate packets arrived at the
appropriate endpoint. The only errors found during testing were found in the
outgoing packets generated by the network. Luckily, these were caused by an
oversight in the code, and can be fixed by setting a value in the appropriate
bus.\\

During development, certain challenges arose in the C\# programming language in
combination with the \gls{sme} framework, primarily missing features and a limited
library with pre-written routines. Other issues and bugs prevented the
networking stack to be compiled and executed on an FPGA, mainly due to lacking
support of C\# \texttt{struct}s or a name-clash in reserved VHDL keywords and
variables in the system. Fortunately, none of these are an issue with the underlying
\gls{sme} model, and they can be fixed in the future.\\

Although the presented networking stack still needs to be rigorously tested and
actually synthesized to a FPGA, we are optimistic that it can bring adequate
performance at a reasonable cost and sensible possibilities for expansion. The
\gls{sme} was an immense improvement for the development compared to more a
traditional \gls{vhdl} approach, and with a few more bug-fixes, improvements,
and enhancements, we are optimistic for the application of the framework.




% - Det virker
%  	Performance (can be good)
%	Easy to extend
%
% - SME
%	Potential, but a lot of work still needs to be done


